%============================= abs.tex================================
\begin{Abstract}

    The current trends in computer architecture are increasingly focusing on
    sharing computing resources among multiple programs and users.
    Multiple programs can share a single core using simultaneous multi-threading
    which is widely supported by most processors and OSes.
    Virtual machine technology allows running multiple OS instances on the same
    processor. While the software and hardware of VMs or multi-threading OS
    is able to isolate illegal access of data to prevent exploits, 
    it cannot prevent the leakage of sensitive data via side-channels
    which exist due to design flaws in shared hardware like caches, branch predictors,
    prefetchers.
    Attackers have successfully been able to extract encryption keys of
    various cryptographically secure algorithms like AES and RSA.
    These leakages are possible and viable because hardware design
    does not take care of the security against such side-channels.
    Moreover, software trojans can use these leakages to create a
    covert channel of communication unknown and undetectable by the OS
    and any software anti-viruses.
    Also, software exploits like return oriented programming
    and buffer overflow attacks can be thwarted more effectively
    with hardware solutions rather than software defenses.
    It has become increasingly necessary to consider data security as
    an important metric for hardware design.

    This thesis first gives a summary of the various side-channel
    attacks using shared hardware which are present in literature.
    The summary gives a motivation for including hardware security as
    an important aspect of hardware design.
    An implementation of a reverse-engineering attack is described
    to extract parameters of the caches present in X86 cores.
    A denial-of-service attack on the prefetcher is described, followed
    by its implementation and test results. A hypothetical side-channel
    using the shared reorder buffer on SMT cores is presented.
%
%
%
%
%
\end{Abstract}
%=======================================================================


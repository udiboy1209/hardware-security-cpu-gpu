%============================= abs.tex================================
\begin{Abstract}
The current trends in computer architecture are increasingly focusing on
sharing computing resources among multiple programs and users.
Multiple programs can share a single core using simultaneous multi-threading
which is widely supported by most of the processors and operating systems.
Virtual machine technology allows running multiple OS instances on the same
processor. While the software and hardware of VMs or multi-threaded OS
is able to isolate illegal access of data to prevent software vulnerabilities, 
it cannot prevent the leakage of sensitive data via side-channels
which exist due to design flaws in shared hardware like caches, branch predictors,
prefetchers.
Attackers have successfully been able to extract encryption keys of
various cryptographically secure algorithms like AES and RSA.
These leakages are possible and viable because hardware design
does not take care of the security against such side-channels.
Moreover, software trojans can use these leakages to create a
covert channel of communication unknown and undetectable by the OS
and any software anti-viruses.
Also, software exploits like return oriented programming
and buffer overflow attacks can be thwarted more effectively
with hardware solutions rather than software defenses.
It has become increasingly necessary to consider data security as
an important metric for hardware design.

An introduction of side-channel attacks is provided as motivation
for including security as an important aspect of hardware design.
We describe how data dependent execution, which is present in AES and RSA ciphers,
can be exploited by different cache side channels like Prime+Probe and Flush+Reload.
As an inital step to cache side channels, we have introduced a method to reverse engineer
cache parameters using microbenchmarking programs.
We propose an attack to disable the prefetcher by preventing
it from generating memory accesses and interfering with side channels
running in the cache.
The attacker is designed to work on a Stride Prefetcher, and is implemented and
tested with OpenSSL AES victim program.
Results show that it is able to significantly reduce the number of prefetches
generated to almost 0.
We also propose a hypothetical side channel which uses the shared Reorder Buffer (ROB) on
SMT cores. This side channel can be used to detect data-dependent stalls in a victim program.
%
%
%
%
%
\end{Abstract}
%=======================================================================


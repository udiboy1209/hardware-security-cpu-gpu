%============================= abs.tex================================
\begin{Abstract}

    The current trends in computer architecture are increasingly 
    focusing on sharing computing resources among multiple programs
    and processes.
    Virtualisation technology and Virtual machines allow different 
    programs to share the same hardware resources.
These shared resources include all the structures inside cores and multi-core processors
which can be accessed simultaneously by threads colocated on a single core, or even processes
on two different cores. This poses a threat to the data security of many critical
processes which run in such a shared context.

Attackers with the right knowledge and tools
can leverage hardware implementation flaws in the design of these shared resources
to extract data from a victim process via undetectable side-channels. Malicious trojans
can use these shared resources to construct covert-channels to establish inter-process
communication undetectable by the core or OS. With the rapid increase of need of
powerful computation resources, GPUs have been extended to support general purpose computing.
More recently, multiple processes are able to share the GPGPU resource
and this opens up a new domain of security attacks which can be mounted on GPGPUs.

With the recent Meltdown \Citeref{meltdown} and Spectre \Citeref{spectre} attacks capable of compromising any Intel core 
regardless of the OS, it is obvious that along with power and performance, design of
computer architecture needs to consider data security as an important metric.
Moreover, a lot of software based attacks like buffer overflow and return-oriented
programming can be thwarted effectively using additional hardware structures.
Hardware support for security against software exploits is an efficient mitigation
and should also be considered when designing processors.\\
\\

In this report, an outline of recent security attack methods and mitigations
on both CPU and GPGPU is presented. The attacks range from side-channels and covert-channels over
various CPU resources to different types of return-oriented programming exploits.
Hardware-based mitigations and their effectiveness over software implementations is shown.
The literature survey is followed with an implementation of cache reverse-engineering procedure
and results.
%
%
%
%
%
\end{Abstract}
%=======================================================================


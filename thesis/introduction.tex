\chapter{Introduction}

In recent years, it has become difficult to keep up with Moore's law using
conventional transistor scaling. Computer architecture has shifted
focus from optimizing single-thread performance to increasing throughput by
running multiple threads and multiple programs simultaneously.
The current trends are increasingly focusing on sharing
computing resources among multiple programs and processes.
Multi-thread and multi-core processors are commonplace in
personal computers and mobile phones, even embedded devices.
In cloud computing, technologies like virtual machines and virtual environments
are allowing multiple different programs to share the same computing resources.
These shared resources include all the structures inside cores and multi-core
processors which can be accessed simultaneously by threads colocated on a single
core, or even processes on two different cores.
This poses a threat to the data security of many critical
processes which run in such a shared context.

Attackers with the right knowledge and tools can leverage hardware
implementation flaws in the design of these shared resources
to extract data from a victim process via undetectable side-channels.
Malicious trojans can use these shared resources to construct
covert-channels to establish inter-process communication undetectable
by the core or OS. With the rapid increase of need of powerful computation
resources, GPUs have been extended to support general purpose computing.
More recently, multiple processes are able to share the GPGPU resource
and this opens up a new domain of security attacks which can be mounted on GPGPUs.

With the recent Meltdown \Citeref{meltdown} and Spectre \Citeref{spectre}
attacks capable of compromising any Intel core regardless of the OS,
it is obvious that along with power and performance, design of
computer architecture needs to consider data security as an important metric.
Moreover, a lot of software based attacks like buffer overflow and return-oriented
programming can be thwarted effectively using additional hardware structures.
Hardware support for security against software exploits is an efficient mitigation
and should also be considered when designing processors.

A lot of side-channel attacks are based on caches due to their common
accessibility between different programs. Cache accesses are abstracted away
from the program hence OS-level access-restrictions do not apply on them.
Caches also have a well-defined memory to cache-line mapping which is used
by the attacker to infer memory addresses. The time-difference between
a cache-hit versus a cache-miss is also noticeable enough to be detected
by an executing program. These characteristics make the cache vulnerable to
side-channel leakages \Citeref{disruptive}.
Cache designs which try to avoid any one of these characteristics lead to severe
performance degradation. For example, if the memory to cache-line mapping is to be
avoided, a fully-associative cache may be used instead of a direct-mapped
or set-associative cache. But a fully-associative design limits the cache-size
to a value much smaller than that desired by modern programs. In fact, the decision
of moving from fully-associative caches to set-associative caches was done
to make larger cache-sizes feasible on modern hardware, and that decision cannot
be undone only for security measures without major impact on performance.

Cache designs like Newcache \Citeref{newcache} try to achieve the same level
of performance while also preventing side-channel leakage.
There are other security methods which add enough noise
to the cache to disrupt any side-channel.
The Disruptive Prefetching \Citeref{disruptive} method utilises
the function of prefetcher to generate random memory accesses to confuse an attacker
while not interfering with program execution and performance.
Another method introduces a new context-sensitive decoder \Citeref{csd}
to mask legitimate memory access instructions with extra random accesses during
instruction decode.

% TODO add prefetch attack
Side channels work best when only the targeted region of code of the victim
is making memory accesses. While it is possible to prevent other programs and
victim's irrelevant code from interfering, hardware which generates memory
accesses like the prefetcher are difficult to stop. The thesis presents an
attack on the prefetcher which tries to completely disable it from
generating any memory accesses. This will help enhance the side channel
to facilitate better and faster key extraction.

% TODO add rob attack
% TODO add ROP attacks

% In this report, an outline of recent security attack methods and mitigations
% on both CPU and GPGPU is presented.
% The attacks range from side-channels and covert-channels over
% various CPU resources to different types of return-oriented programming exploits.
% Hardware-based mitigations and their effectiveness over software
% implementations is shown. The literature survey is followed with an
% implementation of cache reverse-engineering procedure and results.
